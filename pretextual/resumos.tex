% ---
% RESUMOS
% ---

% RESUMO em português
\setlength{\absparsep}{18pt} % ajusta o espaçamento dos parágrafos do resumo
\begin{resumo}
  A avaliação de risco de crédito é essencial para instituições financeiras,
  permitindo a concessão responsável de empréstimos e a redução de
  prejuízos. Com
  o avanço do aprendizado de máquina, modelos preditivos tornaram-se
  ferramentas valiosas para identificar possíveis casos de
  inadimplência, auxiliando
  na tomada de decisões mais precisas e reduzindo drasticamente o
  trabalho manual.
  No entanto, a previsão de inadimplência é desafiada pelo
  desequilíbrio dos conjuntos
  de dados, onde casos de calote são significativamente mais raros do que
  pagamentos bem-sucedidos. Esse desbalanceamento leva a modelos tendenciosos,
  que frequentemente falham em identificar tomadores de alto risco com
  precisão. Para mitigar esse problema, técnicas como re-amostragem de dados e
  aprendizagem sensível ao custo são amplamente utilizadas. Este projeto tem o
  objetivo de realizar uma avaliação comparativa de técnicas para
  lidar com desbalanceamento
  de dados no contexto de avaliação de risco de crédito.

  \textbf{Palavras-chave}: Previsão de inadimplência. Desbalanceamento de
  dados. Aprendizado sensível ao custo. Classes desbalanceadas. Risco
  de crédito.
\end{resumo}

% ABSTRACT in english
\begin{resumo}
  [Abstract]
  \begin{otherlanguage*}
    {english} Credit risk assessment is essential for financial institutions,
    enabling responsible lending and loss reduction. With the advancement of
    machine learning, predictive models have become valuable tools for
    identifying potential default cases, supporting more accurate
    decision-making
    and drastically reducing manual work. However, default prediction
    is challenged
    by the imbalance in datasets, where default cases are significantly
    rarer than successful payments. This imbalance leads to biased models that
    often fail to accurately identify high-risk borrowers. To mitigate this
    issue, techniques such as data resampling and cost-sensitive learning are
    widely used. This project aims to perform a comparative evaluation of
    techniques for handling data imbalance in the context of credit risk
    assessment.

    \vspace{\onelineskip}

    \noindent
    \textbf{Keywords}: Default prediction. Data imbalance.
    Cost-sensitive learning.
    Imbalanced classes. Credit risk.
  \end{otherlanguage*}
\end{resumo}
