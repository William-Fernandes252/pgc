% ----------------------------------------------------------
% Introdução
% Capítulo sem numeração, mas presente no Sumário
% ----------------------------------------------------------

\chapter[Introdução]{Introdução}
\addcontentsline{toc}{chapter}{Introdução}

% Este documento segue as normas estabelecidas
% pela~\citeonline[3.1-3.2]{NBR6028:2003}.

% \section*{Figuras}\label{sec:figuras}
% \addcontentsline{toc}{section}{figuras}

% As normas da~\citeonline[3.1-3.2]{NBR6028:2003} especificam que o
% caption da figura deve vir abaixo da mesma.

% A Figura~\ref{fig:log} ilustra...

% \begin{figure}[htpb]
%    \centering
%    \includegraphics[scale=.3]{figs/logo}
%    \caption{Breve explicação sobre a figura. Deve vir abaixo da mesma.}
%    \label{fig:log}
% \end{figure}

% \section*{Tabelas}\label{sec:tabelas}
% \addcontentsline{toc}{section}{tabelas}

% A Tabela~\ref{tab:tabela} apresenta os resultados...

% \begin{table}[htpb]
%    \centering
%    \caption{Breve explicação sobre a tabela. Deve vir acima da
% mesma.}\label{tab:tabela}
%    \begin{tabular}{|l|c|c|c|c|c|c|r|}
%         \hline
%         \small{XX} & \small{FF} & \small{PP} & \small{YY} &
% \small{Yr} & \small{xY} & \small{Yx} & \small{ZZ} \\ \hline
%                615 &    18      &     2558   &    0,9930  &
% 0,9930  &    0,9930  &    0,9930  &    0,9930  \\ \hline
%                615 &    18      &     2558   &    0,9930  &
% 0,9930  &    0,9930  &    0,9930  &    0,9930  \\ \hline
%                615 &    18      &     2558   &    0,9930  &
% 0,9930  &    0,9930  &    0,9930  &    0,9930  \\ \hline
%                615 &    18      &     2558   &    0,9930  &
% 0,9930  &    0,9930  &    0,9930  &    0,9930  \\ \hline
%                615 &    18      &     2558   &    0,9930  &
% 0,9930  &    0,9930  &    0,9930  &    0,9930  \\ \hline
%    \end{tabular}
% \end{table}

A avaliação de risco de crédito é um componente essencial para a estabilidade do mercado financeiro, pois permite que instituições avaliem com precisão a probabilidade de inadimplência de clientes. Erros nessa avaliação podem resultar não apenas em prejuízos financeiros individuais, mas também em crises sistêmicas, como evidenciado na crise do subprime de 2008.

Nos últimos anos, o uso de algoritmos de aprendizado de máquina tem se consolidado como uma alternativa promissora aos modelos estatísticos tradicionais, como a análise discriminante linear e a regressão logística. Tais algoritmos oferecem maior flexibilidade na modelagem de dados complexos e têm apresentado desempenho superior em diversos cenários de previsão de crédito~\cite{Shi2022}.

No entanto, um desafio persistente nesse domínio é o desbalanceamento das classes: casos de inadimplência são, geralmente, muito menos frequentes do que pagamentos em dia. Esse desequilíbrio tende a enviesar os modelos preditivos, reduzindo sua capacidade de identificar tomadores de alto risco, justamente os casos mais relevantes na prática~\cite{Namvar2018}.

Para lidar com esse problema, diversas estratégias têm sido propostas. Dentre elas, destacam-se as técnicas de re-amostragem, como o SMOTE e o \textit{under-sampling}, que tentam ajustar a distribuição das classes no conjunto de dados; e abordagens de aprendizado sensível ao custo, como o \textit{MetaCost}, que buscam incorporar os impactos assimétricos dos erros de classificação ao próprio processo de aprendizagem~\cite{FernndezCs2018,Wei2025}.

Este trabalho tem como objetivo realizar um estudo comparativo entre diferentes estratégias para lidar com o desbalanceamento de dados na avaliação de risco de crédito. Serão considerados cenários variados, incluindo crédito pessoal, financiamento imobiliário e crédito corporativo, a fim de avaliar a robustez das abordagens analisadas em contextos distintos. Ao final, espera-se oferecer subsídios para a escolha de técnicas mais adequadas à construção de modelos preditivos eficazes em situações de classes desbalanceadas.

\section{Justificativa}\label{sec:motivacao}
\addcontentsline{toc}{section}{Motivação}

\section{Objetivos}\label{sec:objetivos}
\addcontentsline{toc}{section}{Objetivos}
