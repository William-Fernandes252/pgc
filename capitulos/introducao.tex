% ----------------------------------------------------------
% Introdução
% Capítulo sem numeração, mas presente no Sumário
% ----------------------------------------------------------

\chapter[Introdução]{Introdução}
\addcontentsline{toc}{chapter}{Introdução}

% Este documento segue as normas estabelecidas
% pela~\citeonline[3.1-3.2]{NBR6028:2003}.

% \section*{Figuras}\label{sec:figuras}
% \addcontentsline{toc}{section}{figuras}

% As normas da~\citeonline[3.1-3.2]{NBR6028:2003} especificam que o
% caption da figura deve vir abaixo da mesma.

% A Figura~\ref{fig:log} ilustra...

% \begin{figure}[htpb]
%    \centering
%    \includegraphics[scale=.3]{figs/logo}
%    \caption{Breve explicação sobre a figura. Deve vir abaixo da mesma.}
%    \label{fig:log}
% \end{figure}

% \section*{Tabelas}\label{sec:tabelas}
% \addcontentsline{toc}{section}{tabelas}

% A Tabela~\ref{tab:tabela} apresenta os resultados...

% \begin{table}[htpb]
%    \centering
%    \caption{Breve explicação sobre a tabela. Deve vir acima da
% mesma.}\label{tab:tabela}
%    \begin{tabular}{|l|c|c|c|c|c|c|r|}
%         \hline
%         \small{XX} & \small{FF} & \small{PP} & \small{YY} &
% \small{Yr} & \small{xY} & \small{Yx} & \small{ZZ} \\ \hline
%                615 &    18      &     2558   &    0,9930  &
% 0,9930  &    0,9930  &    0,9930  &    0,9930  \\ \hline
%                615 &    18      &     2558   &    0,9930  &
% 0,9930  &    0,9930  &    0,9930  &    0,9930  \\ \hline
%                615 &    18      &     2558   &    0,9930  &
% 0,9930  &    0,9930  &    0,9930  &    0,9930  \\ \hline
%                615 &    18      &     2558   &    0,9930  &
% 0,9930  &    0,9930  &    0,9930  &    0,9930  \\ \hline
%                615 &    18      &     2558   &    0,9930  &
% 0,9930  &    0,9930  &    0,9930  &    0,9930  \\ \hline
%    \end{tabular}
% \end{table}

A avaliação de risco de crédito é um processo crítico para instituições
financeiras, pois envolve a análise da capacidade de um tomador de empréstimo em
honrar suas obrigações financeiras. Essa análise é fundamental para a concessão
de crédito e a gestão do risco associado. Além disso, com os avanços recentes na
pesquisa em ciência de dados, técnicas de aprendizado de máquina têm
sido cada vez mais empregadas na análise de dados financeiros para
auxiliar no processo de tomada de decisão.

\section{Justificativa}\label{sec:motivacao}
\addcontentsline{toc}{section}{Motivação}

% \lipsum[35]

\section{Objetivos}\label{sec:objetivos}
\addcontentsline{toc}{section}{Objetivos}

% \lipsum[36]
