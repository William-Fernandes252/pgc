\chapter{Materiais e Métodos}\label{cap:ferramentas}

Este trabalho compara técnicas para o tratamento de dados desbalanceados no contexto da análise de risco de crédito. Para isso, cada técnica foi utilizada em conjunto com diferentes algoritmos de classificação, aplicados a conjuntos de dados de naturezas diversas. O desempenho dos modelos resultantes foi então avaliado por meio de métricas apropriadas para o problema em questão.

Este capítulo está estruturado da seguinte forma: a Seção~\ref{sec:datasets} apresenta os conjuntos de dados utilizados nos experimentos. A Seção~\ref{sec:configuracao-experimentos} descreve os algoritmos, as técnicas de tratamento de desbalanceamento, os hiper-parâmetros e as metodologias de avaliação de desempenho preditivo dos classificadores.

\section{Conjuntos de Dados}\label{sec:datasets}

Com o objetivo de realizar uma análise abrangente sobre a eficácia das técnicas de tratamento de desbalanceamento, foram selecionados três conjuntos de dados de crédito com características distintas, variando o grau de desequilíbrio entre \(1:9\) e \(1:99\).

O primeiro conjunto de dados refere-se aos registros de crédito da \textit{Lending Club}, uma plataforma norte-americana de empréstimos entre pessoas (\textit{peer-to-peer lending}). A expansão desse mercado resultou em um grande volume de dados transacionais~\cite{Namvar2018}. A empresa disponibiliza publicamente parte desses dados para fins de pesquisa. A análise deste trabalho baseia-se nos registros de 2016 e 2017, que somam aproximadamente \(630.000\) observações e 145 atributos.

O segundo conjunto de dados contém registros de inadimplência de cartões de crédito em Taiwan, referentes ao ano de 2005. A base é composta por aproximadamente 30.000 amostras, incluindo informações demográficas, limites de crédito e históricos de pagamento. Este conjunto de dados é notório pelo seu elevado grau de desbalanceamento~\cite{Wei2025}.

Por fim, o terceiro conjunto de dados é focado em crédito corporativo, relacionando indicadores fundamentalistas de empresas — como margem líquida, retorno sobre o patrimônio líquido, \textit{return on equity} (ROE), e alavancagem financeira — a uma classificação de risco de crédito.

\section{Configuração dos Experimentos}\label{sec:configuracao-experimentos}

Os dados foram segmentados em conjuntos de treino e teste na proporção de \(70\%\) e \(30\%\), respectivamente, uma estratégia consistente com a literatura de referência. Adicionalmente, empregou-se o método de validação cruzada \textit{k-fold} durante a etapa de treinamento para a otimização de hiper-parâmetros. Os algoritmos de classificação selecionados foram \textit{Support Vector Machine} (SVM), \textit{Random Forest} e \textit{AdaBoost}, utilizando suas implementações disponíveis na biblioteca \textit{Scikit-learn}~\cite{Pedregosa2011scikit}.

Neste projeto, comparam-se cinco abordagens para o tratamento de dados desbalanceados, avaliando os modelos resultantes por meio das métricas de acurácia balanceada, \textit{G-mean}, sensibilidade e especificidade. As técnicas de reamostragem escolhidas foram:
\begin{itemize}
  \item \textbf{RUS (Random Under-sampling):} representando as técnicas de subamostragem;
  \item \textbf{SMOTE (Synthetic Minority Over-sampling Technique):} como representante da sobreamostragem;
  \item \textbf{SMOTE-Tomek:} uma abordagem híbrida que combina as duas estratégias anteriores.
\end{itemize}
Para a aprendizagem sensível a custos, foram utilizadas duas abordagens distintas:
\begin{itemize}
  \item \textbf{\textit{MetaCost}:} representando as técnicas de meta-aprendizado;
  \item \textbf{\textit{Cost-sensitive SVM} (CSSVM):} uma implementação do SVM sensível a custos proposta por Iranmehr, Masnadi-Shirazi e Vasconcelos (\citeyear{Iranmehr2019}), como exemplo de abordagem algorítmica.
\end{itemize}

Dada a ausência de uma matriz de custos predefinida para os conjuntos de dados, os custos de classificação incorreta foram tratados como hiperparâmetros. Estes foram otimizados por meio de busca em grade (\textit{grid search}), uma metodologia também adotada por Cao, Zhao e Zaiane (\citeyear{Cao2013}). Por convenção, este trabalho rotula a classe minoritária (inadimplentes) como positiva (1) e a classe majoritária (adimplentes) como negativa (0).